\documentclass[
    title={Lorem Ipsum},
    author={Henry Díaz Bordón},
    publisher={Summer of Math Exposition IV}
]{hessay}

\usepackage{lipsum}

\begin{document}
    \begin{titlepage}
        \begin{tikzpicture}[overlay,remember picture]
            \draw [line width=1.0pt]
                ($ (current page.north west) + (1cm,-1cm) $)
                rectangle
                ($ (current page.south east) + (-1cm,1cm) $);
            \draw [line width=2pt]
                ($ (current page.north west) + (1.2cm,-1.2cm) $)
                rectangle
                ($ (current page.south east) + (-1.2cm,1.2cm) $);
        \end{tikzpicture}
        \begin{center}
            \vspace*{1cm}
            \Huge
            \textbf{\doctitle}
                
            \vspace{1.5cm}

            \LARGE
            \textsc{\docauthor}
                
            \vfill

            \includegraphics[width=0.7\textwidth]{example-image-golden}

            \vfill
                
            \docpublisher
                
            \vspace{0.8cm}
        \end{center}
    \end{titlepage}

    \chapter{La razón áurea y la perfección geométrica}
    \epigraph{El gran libro de la naturaleza está escrito con símbolos matemáticos.}{Galileo Galilei}
    Durante los tiempos de la antigua Grecia, el estudio de las proporciones cautivó a las mentes más brillantes de por entonces. Renombrados filósofos como Pitágoras o Euclides recurrieron a ellas como medio para comprender nuestro mundo, y los artistas celebérrimos de este periodo las consideraron uno de los instrumentos más útiles a su disposición. Las proporciones fueron, de lejos, el artilugio matemático favorito del mundo clásico.

    \lipsum[1-10]

    \nocite{*}
    \printbibliography
\end{document}